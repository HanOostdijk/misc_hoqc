\documentclass[]{article}
\usepackage{lmodern}
\usepackage{amssymb,amsmath}
\usepackage{ifxetex,ifluatex}
\usepackage{fixltx2e} % provides \textsubscript
\ifnum 0\ifxetex 1\fi\ifluatex 1\fi=0 % if pdftex
  \usepackage[T1]{fontenc}
  \usepackage[utf8]{inputenc}
\else % if luatex or xelatex
  \ifxetex
    \usepackage{mathspec}
  \else
    \usepackage{fontspec}
  \fi
  \defaultfontfeatures{Ligatures=TeX,Scale=MatchLowercase}
  \newcommand{\euro}{€}
\fi
% use upquote if available, for straight quotes in verbatim environments
\IfFileExists{upquote.sty}{\usepackage{upquote}}{}
% use microtype if available
\IfFileExists{microtype.sty}{%
\usepackage{microtype}
\UseMicrotypeSet[protrusion]{basicmath} % disable protrusion for tt fonts
}{}
\usepackage[a4paper,portrait,margin=1in]{geometry}
\usepackage{hyperref}
\PassOptionsToPackage{usenames,dvipsnames}{color} % color is loaded by hyperref
\hypersetup{unicode=true,
            pdftitle={Getting location information with the Google Maps API},
            pdfauthor={Han Oostdijk (www.hanoostdijk.nl)},
            colorlinks=true,
            linkcolor=blue,
            citecolor=Blue,
            urlcolor=Blue,
            breaklinks=true}
\urlstyle{same}  % don't use monospace font for urls
\usepackage{color}
\usepackage{fancyvrb}
\newcommand{\VerbBar}{|}
\newcommand{\VERB}{\Verb[commandchars=\\\{\}]}
\DefineVerbatimEnvironment{Highlighting}{Verbatim}{commandchars=\\\{\}}
% Add ',fontsize=\small' for more characters per line
\usepackage{framed}
\definecolor{shadecolor}{RGB}{248,248,248}
\newenvironment{Shaded}{\begin{snugshade}}{\end{snugshade}}
\newcommand{\KeywordTok}[1]{\textcolor[rgb]{0.13,0.29,0.53}{\textbf{{#1}}}}
\newcommand{\DataTypeTok}[1]{\textcolor[rgb]{0.13,0.29,0.53}{{#1}}}
\newcommand{\DecValTok}[1]{\textcolor[rgb]{0.00,0.00,0.81}{{#1}}}
\newcommand{\BaseNTok}[1]{\textcolor[rgb]{0.00,0.00,0.81}{{#1}}}
\newcommand{\FloatTok}[1]{\textcolor[rgb]{0.00,0.00,0.81}{{#1}}}
\newcommand{\ConstantTok}[1]{\textcolor[rgb]{0.00,0.00,0.00}{{#1}}}
\newcommand{\CharTok}[1]{\textcolor[rgb]{0.31,0.60,0.02}{{#1}}}
\newcommand{\SpecialCharTok}[1]{\textcolor[rgb]{0.00,0.00,0.00}{{#1}}}
\newcommand{\StringTok}[1]{\textcolor[rgb]{0.31,0.60,0.02}{{#1}}}
\newcommand{\VerbatimStringTok}[1]{\textcolor[rgb]{0.31,0.60,0.02}{{#1}}}
\newcommand{\SpecialStringTok}[1]{\textcolor[rgb]{0.31,0.60,0.02}{{#1}}}
\newcommand{\ImportTok}[1]{{#1}}
\newcommand{\CommentTok}[1]{\textcolor[rgb]{0.56,0.35,0.01}{\textit{{#1}}}}
\newcommand{\DocumentationTok}[1]{\textcolor[rgb]{0.56,0.35,0.01}{\textbf{\textit{{#1}}}}}
\newcommand{\AnnotationTok}[1]{\textcolor[rgb]{0.56,0.35,0.01}{\textbf{\textit{{#1}}}}}
\newcommand{\CommentVarTok}[1]{\textcolor[rgb]{0.56,0.35,0.01}{\textbf{\textit{{#1}}}}}
\newcommand{\OtherTok}[1]{\textcolor[rgb]{0.56,0.35,0.01}{{#1}}}
\newcommand{\FunctionTok}[1]{\textcolor[rgb]{0.00,0.00,0.00}{{#1}}}
\newcommand{\VariableTok}[1]{\textcolor[rgb]{0.00,0.00,0.00}{{#1}}}
\newcommand{\ControlFlowTok}[1]{\textcolor[rgb]{0.13,0.29,0.53}{\textbf{{#1}}}}
\newcommand{\OperatorTok}[1]{\textcolor[rgb]{0.81,0.36,0.00}{\textbf{{#1}}}}
\newcommand{\BuiltInTok}[1]{{#1}}
\newcommand{\ExtensionTok}[1]{{#1}}
\newcommand{\PreprocessorTok}[1]{\textcolor[rgb]{0.56,0.35,0.01}{\textit{{#1}}}}
\newcommand{\AttributeTok}[1]{\textcolor[rgb]{0.77,0.63,0.00}{{#1}}}
\newcommand{\RegionMarkerTok}[1]{{#1}}
\newcommand{\InformationTok}[1]{\textcolor[rgb]{0.56,0.35,0.01}{\textbf{\textit{{#1}}}}}
\newcommand{\WarningTok}[1]{\textcolor[rgb]{0.56,0.35,0.01}{\textbf{\textit{{#1}}}}}
\newcommand{\AlertTok}[1]{\textcolor[rgb]{0.94,0.16,0.16}{{#1}}}
\newcommand{\ErrorTok}[1]{\textcolor[rgb]{0.64,0.00,0.00}{\textbf{{#1}}}}
\newcommand{\NormalTok}[1]{{#1}}
\usepackage{graphicx,grffile}
\makeatletter
\def\maxwidth{\ifdim\Gin@nat@width>\linewidth\linewidth\else\Gin@nat@width\fi}
\def\maxheight{\ifdim\Gin@nat@height>\textheight\textheight\else\Gin@nat@height\fi}
\makeatother
% Scale images if necessary, so that they will not overflow the page
% margins by default, and it is still possible to overwrite the defaults
% using explicit options in \includegraphics[width, height, ...]{}
\setkeys{Gin}{width=\maxwidth,height=\maxheight,keepaspectratio}
\setlength{\parindent}{0pt}
\setlength{\parskip}{6pt plus 2pt minus 1pt}
\setlength{\emergencystretch}{3em}  % prevent overfull lines
\providecommand{\tightlist}{%
  \setlength{\itemsep}{0pt}\setlength{\parskip}{0pt}}
\setcounter{secnumdepth}{0}

%%% Use protect on footnotes to avoid problems with footnotes in titles
\let\rmarkdownfootnote\footnote%
\def\footnote{\protect\rmarkdownfootnote}

%%% Change title format to be more compact
\usepackage{titling}

% Create subtitle command for use in maketitle
\newcommand{\subtitle}[1]{
  \posttitle{
    \begin{center}\large#1\end{center}
    }
}

\setlength{\droptitle}{-2em}
  \title{Getting location information with the Google Maps API}
  \pretitle{\vspace{\droptitle}\centering\huge}
  \posttitle{\par}
  \author{Han Oostdijk (www.hanoostdijk.nl)}
  \preauthor{\centering\large\emph}
  \postauthor{\par}
  \predate{\centering\large\emph}
  \postdate{\par}
  \date{March 10, 2016}


% hoqc start inserted styles.tex (via yaml 'in_header:')

% definitions related to graphics
\usepackage{subfig}
%\usepackage[margin=10pt,font=small,labelfont=bf,labelsep=endash]{caption}
%\usepackage{graphicx}
%\usepackage{epstopdf}
% ensure sans-serif is the default
\renewcommand*{\familydefault}{\sfdefault}
% define my alias for text macros
\newcommand{\mytextrm}[1]{\textrm{#1}}
\newcommand{\mytextsf}[1]{\textsf{#1}}
\newcommand{\mytexttt}[1]{\texttt{#1}}
\newcommand{\mytextit}[1]{\textit{#1}}
\newcommand{\mytextsl}[1]{\textsl{#1}}
\newcommand{\mytextbf}[1]{\textbf{#1}}
\newcommand{\mytextsc}[1]{\textsc{#1}}
\newcommand{\myemp}[1]{\emp{#1}}
\newcommand{\BSLASH}[1]{\texttt{\char`\\}{#1}}
\newcommand{\mysim}{$\sim$} % use when want to see a tilde in equation 
\newcommand{\mytildev}{\char`\~} % use when want to see a tilde
\newcommand{\mytilde}{~}  % use in rmd so that it survives knitr and pandoc
\newcommand{\mybslash}{ }

% hoqc end inserted styles.tex (via yaml 'in_header:')

% Redefines (sub)paragraphs to behave more like sections
\ifx\paragraph\undefined\else
\let\oldparagraph\paragraph
\renewcommand{\paragraph}[1]{\oldparagraph{#1}\mbox{}}
\fi
\ifx\subparagraph\undefined\else
\let\oldsubparagraph\subparagraph
\renewcommand{\subparagraph}[1]{\oldsubparagraph{#1}\mbox{}}
\fi

\begin{document}
\maketitle

\section{Introduction}\label{introduction}

Google Maps is well known as an interface to view maps. In this document
we use the
\href{https://developers.google.com/maps/web-services/}{Google Maps
Geocoding API} to lookup addresses. For more than incidental work a
\href{https://developers.google.com/maps/documentation/geolocation/usage-limits}{license}
could be necessary.

We will use R to do this but it can also be done in other environments
such as e.g.~javascript, Python or PHP. We show how the API returns an
XML-file as response to an address input. The input can be a full
address or a uniquely identifying part of it. It is assumed that the
XML-file contains at most one result. For more than one result a loop
over the result item would be needed as shown in example 6 in the
\protect\hyperlink{Exampleofuse}{Example of use} section. On special
request this could be done :) .

\section{Used libraries}\label{used-libraries}

\begin{Shaded}
\begin{Highlighting}[]
\KeywordTok{library}\NormalTok{(xml2)}
\KeywordTok{library}\NormalTok{(magrittr)}
\KeywordTok{library}\NormalTok{(xtable)}
\end{Highlighting}
\end{Shaded}

\section{Utility functions}\label{utility-functions}

\subsection{Request an XML document from the
API}\label{request-an-xml-document-from-the-api}

The \mytextbf{read\_address} function actually reads the information
from the Google website. It returns an XML-document. When you want to
see the contents of the document, uncomment the line with
\mytextbf{write\_xml}.

\begin{Shaded}
\begin{Highlighting}[]
\NormalTok{read_address <-function(address) \{}
  \NormalTok{url <-}\StringTok{ }\KeywordTok{paste0}\NormalTok{(}\StringTok{'http://maps.google.com/maps/api/geocode/xml'}\NormalTok{,}
                    \StringTok{'?sensor=false&address='}\NormalTok{,}\KeywordTok{url_escape}\NormalTok{(address))}
  \NormalTok{doc <-}\StringTok{ }\KeywordTok{read_xml}\NormalTok{(url)}
  \CommentTok{# write_xml(doc, 'temp.xml')}
  \KeywordTok{return}\NormalTok{(doc)}
\NormalTok{\}}
\end{Highlighting}
\end{Shaded}

\subsection{Get type(s) from document}\label{get-types-from-document}

The XML document contains elements that describe an attribute of the
address such as the country, administrative area (in the Netherlands a
province) or location. The \mytextit{type} tag indicates which attribute
it concerns. The \mytextbf{address\_data} function select a list of
types from the document and uses the \mytextbf{get\_type} function for
each of these. For some addresssen not all types are present: e.g.~when
the address is a country, the postcode is not present. The line with
\mytextit{gsub} handles that. \mytextit{Types} is a list with in the
first element the types that are collected and in the second one the
names that they will be assigned.

\begin{Shaded}
\begin{Highlighting}[]
\NormalTok{get_type <-function(tt,type_nodes) \{}
  \NormalTok{z=}\KeywordTok{grepl}\NormalTok{(tt,}\KeywordTok{sapply}\NormalTok{(type_nodes,xml_text))}
  \KeywordTok{xml_text}\NormalTok{(}\KeywordTok{xml_find_all}\NormalTok{(}\KeywordTok{xml_parent}\NormalTok{(type_nodes[z]),}\StringTok{"./long_name"}\NormalTok{))}
\NormalTok{\}}
\NormalTok{address_data <-function(doc,types) \{}
  \NormalTok{type_nodes =}\StringTok{ }\KeywordTok{xml_find_all}\NormalTok{(doc, }\StringTok{".//type"}\NormalTok{)}
  \NormalTok{w=}\KeywordTok{sapply}\NormalTok{(types$i,function(x) \{}\KeywordTok{get_type}\NormalTok{(x,type_nodes)\})}
  \NormalTok{w=}\KeywordTok{as.character}\NormalTok{(w); w=}\KeywordTok{gsub}\NormalTok{(}\StringTok{"character(0)"}\NormalTok{,}\StringTok{""}\NormalTok{,w,}\DataTypeTok{fixed=}\NormalTok{T)}
  \KeywordTok{names}\NormalTok{(w)=types$o}
  \KeywordTok{return}\NormalTok{(w)}
\NormalTok{\}}
\end{Highlighting}
\end{Shaded}

\subsection{Get coordinates from
document}\label{get-coordinates-from-document}

The coordinates are given in \mytextbf{lat} (latitude or
\mytextit{breedte} in Dutch) and \mytextbf{long} (longitude or
\mytextit{lengte}). The document contains (in my experience) three sets
of these pairs. The first set is found in the element
\mytextit{location} and indicates the coordinates of the centre of the
address. The second set is found in the element
\mytextit{viewport/southwest} and indicates the coordinates of the
southwest corner of the address and the third one indicates the
northeast corner of the address. The \mytextbf{tc} parameter indicates
the element and the \mytextbf{n} parameter the names these coordinates
will get.

\begin{Shaded}
\begin{Highlighting}[]
\NormalTok{coord_data <-function(doc,}\DataTypeTok{tc=}\StringTok{'location'}\NormalTok{,}\DataTypeTok{n=}\KeywordTok{c}\NormalTok{(}\StringTok{'lat'}\NormalTok{,}\StringTok{'lng'}\NormalTok{)) \{}
  \NormalTok{loc =}\StringTok{ }\KeywordTok{xml_find_all}\NormalTok{(doc, }\KeywordTok{paste0}\NormalTok{(}\StringTok{".//"}\NormalTok{,tc))}
  \NormalTok{lat =}\StringTok{ }\NormalTok{loc %>%}\StringTok{ }\KeywordTok{xml_find_all}\NormalTok{(}\StringTok{".//lat"}\NormalTok{) %>%}\StringTok{ }\KeywordTok{xml_text}\NormalTok{() }
  \NormalTok{lng =}\StringTok{ }\NormalTok{loc %>%}\StringTok{ }\KeywordTok{xml_find_all}\NormalTok{(}\StringTok{".//lng"}\NormalTok{) %>%}\StringTok{ }\KeywordTok{xml_text}\NormalTok{() }
  \NormalTok{w   =}\StringTok{ }\KeywordTok{c}\NormalTok{(lat,lng)}
  \KeywordTok{names}\NormalTok{(w)=n}
  \NormalTok{w}
\NormalTok{\}}
\end{Highlighting}
\end{Shaded}

\subsection{Print data.frame}\label{print-data.frame}

The function \mytextbf{print\_adress} prints a data.frame . In this
example it is used to print the results of the examples. Apart from the
data.frames there are parameters to indicate the \LaTeX~label and
caption and the number of digits that will be shown.

\begin{Shaded}
\begin{Highlighting}[]
\NormalTok{def_tab <-}\StringTok{ }\NormalTok{function (label_name,label_tekst)  \{ }
  \KeywordTok{paste0}\NormalTok{(label_tekst,}\StringTok{"}\CharTok{\textbackslash{}\textbackslash{}}\StringTok{label\{table:"}\NormalTok{,label_name,}\StringTok{"\}"}\NormalTok{)}
\NormalTok{\}}

\NormalTok{print_adress <-}\StringTok{ }\NormalTok{function (df,lbl,cap,}\DataTypeTok{digits=}\DecValTok{3}\NormalTok{) \{}
  \KeywordTok{print}\NormalTok{(}\KeywordTok{xtable}\NormalTok{(df,}\DataTypeTok{caption=}\KeywordTok{def_tab}\NormalTok{(lbl,cap),}\DataTypeTok{digits=}\NormalTok{digits), }
    \DataTypeTok{rownames=}\NormalTok{F, }\DataTypeTok{table.placement=}\StringTok{'!htbp'}\NormalTok{)}
\NormalTok{\}}
\end{Highlighting}
\end{Shaded}

\section{Get all data for an address}\label{get-all-data-for-an-address}

The \mytextbf{all\_data} function combines the utility functions and
creates a character vector with the attributes of the address and all
its coordinates. When the API does not return results (maybe an invalid
address was specified) the function does not return a vector but the
boolean value FALSE.

\begin{Shaded}
\begin{Highlighting}[]
\NormalTok{all_data <-}\StringTok{ }\NormalTok{function(address,}\DataTypeTok{n=}\KeywordTok{c}\NormalTok{(}\StringTok{'lat'}\NormalTok{,}\StringTok{'lng'}\NormalTok{)) \{}
  \NormalTok{doc =}\StringTok{ }\KeywordTok{read_address}\NormalTok{(address)}
  \NormalTok{if (}\KeywordTok{xml_find_all}\NormalTok{(doc, }\StringTok{".//status"}\NormalTok{) %>%}\StringTok{ }\KeywordTok{xml_text}\NormalTok{() ==}\StringTok{ "ZERO_RESULTS"}\NormalTok{) \{}
    \KeywordTok{return}\NormalTok{(F)}
  \NormalTok{\}}
  \NormalTok{types =}\StringTok{ }\KeywordTok{list}\NormalTok{(}\DataTypeTok{i=}\KeywordTok{c}\NormalTok{(}\StringTok{'postal_code'}\NormalTok{,}\StringTok{'locality'}\NormalTok{,}
                      \StringTok{'administrative_area_level_1'}\NormalTok{,}\StringTok{'country'}\NormalTok{),}
               \DataTypeTok{o=}\KeywordTok{c}\NormalTok{(}\StringTok{'postcode'}\NormalTok{,}\StringTok{'location'}\NormalTok{,}\StringTok{'level2'}\NormalTok{,}\StringTok{'country'}\NormalTok{) )}
  \NormalTok{a1=}\KeywordTok{address_data}\NormalTok{(doc,types)}
  \NormalTok{c1=}\KeywordTok{coord_data}\NormalTok{(doc,}\DataTypeTok{n=}\NormalTok{n)}
  \NormalTok{c2=}\KeywordTok{coord_data}\NormalTok{(doc,}\StringTok{'viewport/southwest'}\NormalTok{,}\DataTypeTok{n=}\KeywordTok{paste0}\NormalTok{(}\StringTok{'sw_'}\NormalTok{,n))}
  \NormalTok{c3=}\KeywordTok{coord_data}\NormalTok{(doc,}\StringTok{'viewport/northeast'}\NormalTok{,}\DataTypeTok{n=}\KeywordTok{paste0}\NormalTok{(}\StringTok{'ne_'}\NormalTok{,n))}
  \KeywordTok{c}\NormalTok{(a1,c1,c2,c3)}
\NormalTok{\}}
\end{Highlighting}
\end{Shaded}

\hypertarget{Exampleofuse}{\section{Example of use}\label{Exampleofuse}}

In this example we give the results from 6 different calls to the API on
page \pageref{table:e1}:

\begin{itemize}
\tightlist
\item
  with the full address in Dutch. Results in Table \ref{table:e1}. NB:
  the country indication is in English.
\item
  with only the numeric part of the postcode and the country indication
  \mytextit{NL}. Results in Table \ref{table:e2}.
\item
  with only the location (\mytextit{Amstelveen}) and the country
  indication \mytextit{NL}. Results in Table \ref{table:e3}.
\item
  with only the country (\mytextit{Nederland}) and the country
  indication \mytextit{NL}. Results in Table \ref{table:e4}.
\item
  with (what I thought to be) an invalid address. Apparently the API
  links this to the country \mytextit{Italy}. Results in Table
  \ref{table:e5}.
\item
  with only the country (\mytextit{Nederland}) and without the country
  indication. In the introduction we already said that this code
  currently works only when there is at most one result. The result of
  \mytextit{print(e6)} shows garbled output.
\end{itemize}

\begin{Shaded}
\begin{Highlighting}[]
\NormalTok{e1=}\KeywordTok{all_data}\NormalTok{(}\StringTok{'Runmoolen 24 Amstelveen Nederland'}\NormalTok{) }
\NormalTok{e2=}\KeywordTok{all_data}\NormalTok{(}\StringTok{'1181 NL'}\NormalTok{) }
\NormalTok{e3=}\KeywordTok{all_data}\NormalTok{(}\StringTok{'Amstelveen NL'}\NormalTok{) }
\NormalTok{e4=}\KeywordTok{all_data}\NormalTok{(}\StringTok{'Nederland NL'}\NormalTok{) }
\NormalTok{e5=}\KeywordTok{all_data}\NormalTok{(}\StringTok{'X Y Z'}\NormalTok{)}
\NormalTok{e6=}\KeywordTok{all_data}\NormalTok{(}\StringTok{'Nederland'}\NormalTok{)}
\KeywordTok{print}\NormalTok{(e6)}
\end{Highlighting}
\end{Shaded}

\begin{verbatim}
                                              postcode 
                                               "80466" 
                                              location 
                     "c(\"Nederland\", \"Nederland\")" 
                                                level2 
                          "c(\"Texas\", \"Colorado\")" 
                                               country 
\end{verbatim}

``c("Netherlands", "United States", "United States")'' lat
``52.1326330'' lng ``29.9743803'' ``39.9613759'' ``5.2912660''
``-93.9923965'' ``-105.5108312'' sw\_lat ``50.7503837'' sw\_lng
``29.9465649'' ``39.9549809'' ``3.3316000'' ``-94.0419501''
``-105.5233440'' ne\_lat ``53.6756000'' ne\_lng ``30.0189810''
``39.9753149'' ``7.2271405'' ``-93.9629590'' ``-105.4841760''

\subsection{Convert to data.frame}\label{convert-to-data.frame}

We convert the vectors in e1 \ldots{}e5 to a data.frame for a better
presentation.

\begin{Shaded}
\begin{Highlighting}[]
\NormalTok{e =}\StringTok{ }\KeywordTok{matrix}\NormalTok{(}\KeywordTok{c}\NormalTok{(e1,e2,e3,e4,e5),}\DataTypeTok{nrow=}\DecValTok{5}\NormalTok{,}\DataTypeTok{byrow=}\NormalTok{T)}
\NormalTok{e =}\StringTok{ }\KeywordTok{data.frame}\NormalTok{(}\DataTypeTok{pc=}\NormalTok{e[,}\DecValTok{1}\NormalTok{],}\DataTypeTok{loc=}\NormalTok{e[,}\DecValTok{2}\NormalTok{],}\DataTypeTok{prov=}\NormalTok{e[,}\DecValTok{3}\NormalTok{],}\DataTypeTok{cntr=}\NormalTok{e[,}\DecValTok{4}\NormalTok{],}
        \DataTypeTok{lat=}\KeywordTok{as.numeric}\NormalTok{(e[,}\DecValTok{5}\NormalTok{]),   }\DataTypeTok{long=}\KeywordTok{as.numeric}\NormalTok{(e[,}\DecValTok{6}\NormalTok{]),}
        \DataTypeTok{slat=}\KeywordTok{as.numeric}\NormalTok{(e[,}\DecValTok{7}\NormalTok{]), }\DataTypeTok{wlong=}\KeywordTok{as.numeric}\NormalTok{(e[,}\DecValTok{8}\NormalTok{]),}
        \DataTypeTok{nlat=}\KeywordTok{as.numeric}\NormalTok{(e[,}\DecValTok{9}\NormalTok{]), }\DataTypeTok{elong=}\KeywordTok{as.numeric}\NormalTok{(e[,}\DecValTok{10}\NormalTok{])   )}
\end{Highlighting}
\end{Shaded}

\subsection{Print the rows of the data.frame
(separately)}\label{print-the-rows-of-the-data.frame-separately}

By printing the rows separately we can give them individually a caption
for better readability.

\begin{Shaded}
\begin{Highlighting}[]
\KeywordTok{print_adress}\NormalTok{(e[}\DecValTok{1}\NormalTok{,],}\StringTok{'e1'}\NormalTok{,cape1,}\DataTypeTok{digits=}\DecValTok{2}\NormalTok{)}
\KeywordTok{print_adress}\NormalTok{(e[}\DecValTok{2}\NormalTok{,],}\StringTok{'e2'}\NormalTok{,cape2,}\DataTypeTok{digits=}\DecValTok{2}\NormalTok{)}
\KeywordTok{print_adress}\NormalTok{(e[}\DecValTok{3}\NormalTok{,],}\StringTok{'e3'}\NormalTok{,cape3,}\DataTypeTok{digits=}\DecValTok{2}\NormalTok{)}
\KeywordTok{print_adress}\NormalTok{(e[}\DecValTok{4}\NormalTok{,],}\StringTok{'e4'}\NormalTok{,cape4,}\DataTypeTok{digits=}\DecValTok{2}\NormalTok{)}
\KeywordTok{print_adress}\NormalTok{(e[}\DecValTok{5}\NormalTok{,],}\StringTok{'e5'}\NormalTok{,cape5,}\DataTypeTok{digits=}\DecValTok{2}\NormalTok{)}
\end{Highlighting}
\end{Shaded}

\begin{table}[!htbp]
\centering
\begin{tabular}{rllllrrrrrr}
  \hline
 & pc & loc & prov & cntr & lat & long & slat & wlong & nlat & elong \\ 
  \hline
1 & 1181 NZ & Amstelveen & Noord-Holland & Netherlands & 52.31 & 4.87 & 52.31 & 4.86 & 52.31 & 4.87 \\ 
   \hline
\end{tabular}
\caption{results for address 'Runmoolen 24 Amstelveen Nederland'\label{table:e1}} 
\end{table}\begin{table}[!htbp]
\centering
\begin{tabular}{rllllrrrrrr}
  \hline
 & pc & loc & prov & cntr & lat & long & slat & wlong & nlat & elong \\ 
  \hline
2 & 1181 & Amstelveen & Noord-Holland & Netherlands & 52.31 & 4.86 & 52.30 & 4.85 & 52.32 & 4.87 \\ 
   \hline
\end{tabular}
\caption{results for address '1181 NL'\label{table:e2}} 
\end{table}\begin{table}[!htbp]
\centering
\begin{tabular}{rllllrrrrrr}
  \hline
 & pc & loc & prov & cntr & lat & long & slat & wlong & nlat & elong \\ 
  \hline
3 &  & Amstelveen & Noord-Holland & Netherlands & 52.31 & 4.87 & 52.24 & 4.79 & 52.33 & 4.91 \\ 
   \hline
\end{tabular}
\caption{results for address 'Amstelveen NL'\label{table:e3}} 
\end{table}\begin{table}[!htbp]
\centering
\begin{tabular}{rllllrrrrrr}
  \hline
 & pc & loc & prov & cntr & lat & long & slat & wlong & nlat & elong \\ 
  \hline
4 &  &  &  & Netherlands & 52.13 & 5.29 & 50.75 & 3.33 & 53.68 & 7.23 \\ 
   \hline
\end{tabular}
\caption{results for address 'Nederland NL'\label{table:e4}} 
\end{table}\begin{table}[!htbp]
\centering
\begin{tabular}{rllllrrrrrr}
  \hline
 & pc & loc & prov & cntr & lat & long & slat & wlong & nlat & elong \\ 
  \hline
5 &  &  &  & Italy & 41.87 & 12.57 & 35.49 & 6.63 & 47.09 & 18.52 \\ 
   \hline
\end{tabular}
\caption{results for address 'X Y Z'\label{table:e5}} 
\end{table}

\clearpage

\section{Session Info}\label{session-info}

\begin{Shaded}
\begin{Highlighting}[]
\KeywordTok{sessionInfo}\NormalTok{()}
\end{Highlighting}
\end{Shaded}

\begin{verbatim}
## R version 3.2.4 (2016-03-10)
## Platform: x86_64-w64-mingw32/x64 (64-bit)
## Running under: Windows 10 x64 (build 10586)
## 
## locale:
## [1] LC_COLLATE=English_United States.1252  LC_CTYPE=English_United States.1252   
## [3] LC_MONETARY=English_United States.1252 LC_NUMERIC=C                          
## [5] LC_TIME=English_United States.1252    
## 
## attached base packages:
## [1] stats     graphics  grDevices utils     datasets  methods   base     
## 
## other attached packages:
## [1] knitr_1.12.3 xtable_1.8-2 magrittr_1.5 xml2_0.1.2  
## 
## loaded via a namespace (and not attached):
##  [1] formatR_1.3     tools_3.2.4     htmltools_0.3   yaml_2.1.13     Rcpp_0.12.3    
##  [6] stringi_1.0-1   rmarkdown_0.9.5 stringr_1.0.0   digest_0.6.9    evaluate_0.8.3
\end{verbatim}

\end{document}
